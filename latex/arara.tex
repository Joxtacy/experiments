% arara: pdflatex
% arara: pdflatex if found('log', 'undefined references')
\documentclass[12pt, a4paper]{article}

\usepackage{graphicx} % LaTeX package to import graphics
\graphicspath{{images/}} % configuring the graphicx package

\usepackage{amsmath} % For the equation* environment

\title{My first LaTeX document}
\author{Jesper Hasselquist\thanks{Funded by the Overleaf team.}}
\date{\today}

\begin{document}
\maketitle
\section{Termite recipes}
\label{sec:termiterecipes}
Hello, this is Section~\ref{sec:termiterecipes}.
Some of the \textbf{greatest}
discoveries in \underline{science} 
were made by \textbf{\textit{accident}}.

Some of the greatest \emph{discoveries} in science 
were made by accident.

\textit{Some of the greatest \emph{discoveries} 
in science were made by accident.}

\textbf{Some of the greatest \emph{discoveries} 
in science were made by accident.}

The universe is immense and it seems to be homogeneous,
on a large scale, everywhere we look.

% The \includegraphics command is
% provided (implemented) by the
% graphicx package
\includegraphics[width=0.75\textwidth]{galaxy}

There's a picture of a galaxy above.

\begin{figure}[h]
    \centering
    \includegraphics[width=0.75\textwidth]{galaxy}
    \caption{A nice plot.}
    \label{fig:mesh1}
\end{figure}

As you \textbf{cannot} see in figure \ref{fig:mesh1}, the function grows near the origin. This example is on page \pageref{fig:mesh1}.

\begin{itemize}
  \item The individual entries are indicated with a black dot, a so-called bullet.
  \item The text in the entries may be of any length.
\end{itemize}

\begin{enumerate}
  \item This is the first entry in our list.
  \item The list numbers increase with each entry we add.
\end{enumerate}

In physics, the mass-energy equivalence is stated 
by the equation $E=mc^2$, discovered in 1905 by Albert Einstein.

\begin{math}
E=mc^2
\end{math} is typeset in a paragraph using inline math mode---as is $E=mc^2$, and so too is \(E=mc^2\).

The mass-energy equivalence is described by the famous equation
\[ E=mc^2 \] discovered in 1905 by Albert Einstein. 

In natural units ($c = 1$), the formula expresses the identity
\begin{equation}
E=m
\end{equation}

Subscripts in math mode are written as $a_b$ and superscripts are written as $a^b$. These can be combined and nested to write expressions such as

\[ T^{i_1 i_2 \dots i_p}_{j_1 j_2 \dots j_q} = T(x^{i_1},\dots,x^{i_p},e_{j_1},\dots,e_{j_q}) \]
 
We write integrals using $\int$ and fractions using $\frac{a}{b}$. Limits are placed on integrals using superscripts and subscripts:

\[ \int_0^1 \frac{dx}{e^x} = \frac{e-1}{e} \]

Lower case Greek letters are written as $\omega$ $\delta$ etc. while upper case Greek letters are written as $\Omega$ $\Delta$.

Mathematical operators are prefixed with a backslash as $\sin(\beta)$, $\cos(\alpha)$, $\log(x)$ etc.

\section{First example}

The well-known Pythagorean theorem \(x^2 + y^2 = z^2\) was proved to be invalid for other exponents, meaning the next equation has no integer solutions for \(n>2\):

\[ x^n + y^n = z^n \]

\section{Second example}

This is a simple math expression \(\sqrt{x^2+1}\) inside text. 
And this is also the same: 
\begin{math}
\sqrt{x^2+1}
\end{math}
but by using another command.

This is a simple math expression without numbering
\[\sqrt{x^2+1}\] 
separated from text.

This is also the same:
\begin{displaymath}
\sqrt{x^2+1}
\end{displaymath}

\ldots and this:
\begin{equation*}
\sqrt{x^2+1}
\end{equation*}
\end{document}
